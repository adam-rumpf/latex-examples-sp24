%========================================================================
% Preamble (General)
%========================================================================

\documentclass{beamer} % Beamer is the document class for slideshows
\usepackage{amsmath,amsfonts,amsthm,amssymb,mathrsfs} % Common math symbols
\usepackage{graphicx,float} % Figure inclusion and placement
\usepackage[scaled=0.92]{helvet} % Font selection
\usepackage{xcolor} % Color selection

%========================================================================
% Preamble (Beamer-Related)
%========================================================================

% -------------------------------
% Select a theme
% -------------------------------
%\usetheme{Warsaw}
%\usetheme{Rochester}
%\usetheme{Madrid}
%\usetheme{Pittsburgh}
%\usetheme{Antibes}
%\usetheme{Montpellier}
%\usetheme{Berkeley}
%\usetheme{PaloAlto}
%\usetheme{Goettingen}
%\usetheme{Marburg}
%\usetheme{Hannover}
\usetheme{Berlin}
%\usetheme{Ilmenau}
%\usetheme{Dresden}
%\usetheme{Darmstadt}
%\usetheme{Frankfurt}
%\usetheme{Singapore}
%\usetheme{Szeged}
%\usetheme{Copenhagen}
%\usetheme{Malmoe}

% -------------------------------
% Select a color theme
% -------------------------------
%\usecolortheme{seagull}
%\usecolortheme{crane}
%\usecolortheme{default}
\usecolortheme[rgb={0.0902,0.404,0.659}]{structure} % Custom blue color

% -------------------------------
 % Select a font theme
% -------------------------------
\usefonttheme{structurebold}
%\usefonttheme{structuresmallcapsserif}
%\usefonttheme{structureitalicserif}
%\usefonttheme{serif}
\usefonttheme[onlymath]{serif}

% -------------------------------
 % Select a background color
% -------------------------------
%\setbeamertemplate{background canvas}[vertical shading][bottom=white,top=gray!30]
%\setbeamertemplate{background canvas}[vertical shading][bottom=white,top=red!10!black!30]
%\setbeamertemplate{background canvas}[vertical shading][bottom=white,top=green!20!black!30]
%\setbeamertemplate{background canvas}[vertical shading][bottom=white,top=white]

% -------------------------------
% Select a color for math text
% -------------------------------
%\setbeamercolor{math text}{fg=red!80!black}
% This command suppresses the navigation symbols at footline
% comment the command below if you  want navigation symbols
%\setbeamertemplate{navigation symbols}{}

% -------------------------------
% A macro to automatically generate section title slides
% -------------------------------
\AtBeginSection[]{
\begin{frame}
\vfill
\centering
%\begin{beamercolorbox}[sep=8pt,center,shadow=true,rounded=true]{title}
\begin{beamercolorbox}[sep=8pt,center,shadow=false,rounded=false]{title}
%\usebeamerfont{title}\insertsectionhead\par%
\usebeamerfont{title}\insertsection\par%
\end{beamercolorbox}
\vfill
\end{frame}
}

%========================================================================
% Preamble (Title Page)
%========================================================================

% Common title strings
\title{Writing Papers Like a Grown-Up:\\ A Crash Course in \LaTeX}
\author{Adam Rumpf\\{\footnotesize \href{mailto:arumpf@floridapoly.edu}{arumpf@floridapoly.edu}}}
\institute{Florida Polytechnic University\\Department of Applied Mathematics}
\date{February 26, 2024}

% Custom footer
\setbeamertemplate{footline}[text line]{\textcolor{gray}{\parbox{\linewidth}{\vspace*{-8pt}Adam Rumpf \hfill \url{adam-rumpf.github.io}}}}
\setbeamertemplate{navigation symbols}{}

%========================================================================
% Slideshow
%========================================================================

\begin{document}

%========================================================================
\begin{frame}

\titlepage

\end{frame}

%========================================================================
\begin{frame}{Outline}

\begin{enumerate}
	\item Introduction to \LaTeX
	\begin{itemize}
		\item What it is
		\item How it works
		\item Why you should use it
	\end{itemize}
	\item A crash course in using \LaTeX
	\begin{itemize}
		\item Minimal working example
		\item A tour of some features
	\end{itemize}
	\item Helpful resources
\end{enumerate}

\end{frame}

%========================================================================
\begin{frame}{Preamble}

\begin{itemize}
	\item Like any programming language, using \LaTeX\ is a skill that takes practice
	\item You cannot learn it from just watching one presentation
\end{itemize}

\end{frame}

%========================================================================
\begin{frame}{Preamble}

\begin{itemize}
	\item All I can do is to show you its basic format and a few features
	\item You will be able to play around with it on your own later
	\item My main goal is to show you some of what is possible so that you can go out to research how to do it on your own later
\end{itemize}

\end{frame}

%========================================================================
%========================================================================
%========================================================================
\section{Introduction}

%========================================================================
\begin{frame}{What is \LaTeX?}

\begin{itemize}
	\item \LaTeX\ is a document typesetting system
	\item It is used extensively in research and throughout academia since it offers a number of advantages over simpler alternatives like \textit{Microsoft Word}
	\item This is particularly true in STEM fields since \LaTeX\ is excellent for formatting mathematical text
\end{itemize}

\end{frame}

%========================================================================
\begin{frame}{What is \LaTeX?}

\begin{itemize}
	\item Unlike a \textit{Word} document, in \LaTeX\ you do not directly edit text through a GUI
	\item \LaTeX\ is a markup language (like HTML)
	\item Everything you type is included in a plain text file (with the \texttt{.tex} extension)
	\item You include tags and other code to control formatting and layout and to produce special symbols
\end{itemize}

\end{frame}

%========================================================================
\begin{frame}{What is \LaTeX?}

You can find today's example code at my \textit{LaTeX Examples} GitHub repo \href{https://github.com/adam-rumpf/latex-examples-sp24}{here}.

\vfill

\scriptsize

HTML: \texttt{<p>You can find today's example code at my <i>LaTeX Examples</i> GitHub repo <a~href="https://github.com/adam-rumpf/latex-examples-sp24">here</a>.</p>}

\quad

\LaTeX: \texttt{You can find today's example code at my \textbackslash textit\{LaTeX Examples\} GitHub repo \textbackslash href\{https://github.com/adam-rumpf/latex-examples-sp24\}\{here\}.}

\end{frame}

%========================================================================
\begin{frame}{What is \LaTeX?}

\begin{figure}
	\begin{minipage}{0.45\textwidth}
		You type \TeX\ code like this:
		
		\quad
		
		{\tt \tiny
		According to \textbackslash textit\{Newton's Law of Cooling\},
		
		\quad
		
		\textbackslash begin\{align\}
		
		\qquad \textbackslash frac\{du\}\{dt\} = k (A - u)
		
		\textbackslash end\{align\}
		
		for \$k > 0\$, the rate of temperature increase is proportional to the temperature difference.
		}
	\end{minipage}
	\qquad
	\begin{minipage}{0.45\textwidth}
		The \LaTeX\ compiler then produces something like this:
		
		\quad
		
		{\tiny
		According to \textit{Newton's Law of Cooling},
		\begin{align}
			\frac{du}{dt} = k (A - u)
		\end{align}
		for $k > 0$, the rate of temperature increase is proportional to the temperature difference.
		}
	\end{minipage}
\end{figure}

\end{frame}

%========================================================================
\begin{frame}{What is \LaTeX?}

\begin{itemize}
	\item You can download a \LaTeX\ for yourself, but using it requires a couple of different components:
	\begin{itemize}
		\item A package manager (e.g.~MiKTeX or TeX Live) to download, update, and access the packages that extend basic \LaTeX\ to do all the fancy things you want
		\item An IDE/compiler (e.g.~TeXworks) to compile \TeX\ code into the final document (usually a PDF)
	\end{itemize}
\end{itemize}

\end{frame}

%========================================================================
\begin{frame}{What is \LaTeX?}

\begin{itemize}
	\item As an easier alternative, most students prefer to begin by using \href{https://www.overleaf.com/}{\textit{Overleaf.com}}
	\item It is a free browser-based \LaTeX\ distribution that also offers limited collaboration
\end{itemize}

\end{frame}

%========================================================================
\begin{frame}{Why is \LaTeX?}

\begin{itemize}
	\item \LaTeX\ is the standard typesetting language used in academia and for technical writing
	\item Many journals and online repositories require submissions in the form of \TeX\ files
	\item There are several reasons for this
\end{itemize}

\end{frame}

%========================================================================
\begin{frame}{Why is \LaTeX?}

\begin{itemize}
	\item For one, the style of the entire document (margins, spacing, text size, font, layout, numbering systems,~etc.) is controlled by the commands at the beginning of the \TeX\ file, called the \textit{preamble}
	
	\quad
	
	{\tt \textbackslash pdfoutput=1
	
	\textbackslash documentclass[11pt]\{article\}
	
	\textbackslash usepackage[margin=1.0in]\{geometry\}
	
	\textbackslash usepackage[numbers,sort\&compress]\{natbib\}}
	
	\quad
	
	\item Changing a bit of code in the preamble can instantly change the entire document
\end{itemize}

\end{frame}

%========================================================================
\begin{frame}{Why is \LaTeX?}

\begin{itemize}
	\item This is useful for collecting submissions from many authors and putting them all into a standard format (like articles in a journal)
	\item It's also useful for copying all or part of a paper into different distributions (articles, posters, books,~etc.)
\end{itemize}

\end{frame}

%========================================================================
\begin{frame}{Why is \LaTeX?}

\begin{itemize}
	\item In case you couldn't tell, this slideshow was made in \LaTeX
	\item There is a document class called \textit{Beamer} for creating slideshows
	\item Because it's still all \LaTeX, formulas and things can be copied directly from papers
\end{itemize}

\end{frame}

%========================================================================
\begin{frame}{Why is \LaTeX?}

\begin{itemize}
	\item For two, \LaTeX\ makes many common tasks in technical writing far easier than they would be in \textit{Word}
	\item Some examples include:
	\begin{itemize}
		\item Typesetting mathematical text, either inline or as a long system of aligned equations
		\item Automatically updating reference numbers (to sections, equations, figures, tables, references,~etc.) when their placement in the document changes
		\item Automatically generating citations (via \textsc{Bib}\TeX)
		\item Directly importing and displaying computer code (via \href{https://ctan.org/pkg/listings}{Listings})
	\end{itemize}
\end{itemize}

\end{frame}

%========================================================================
\begin{frame}{Why is \LaTeX?}

\begin{itemize}
	\item The \LaTeX\ commands for mathematical typesetting are very commonly used in other pieces of software (e.g.~MATLAB)
	\item e.g. $$\int_a^b f(x) \, dx := \lim_{n \to \infty} \sum_{i=1}^n f(x_i^*) \, \Delta x$$
\end{itemize}

\scriptsize

\texttt{\textbackslash int\_a\^{}b f(x) dx := \textbackslash lim\_\{n \textbackslash to \textbackslash infty\} \textbackslash sum\_\{i=1\}\^{}n f(x\_i\^{}*) \textbackslash Delta x}

\end{frame}

%========================================================================
\begin{frame}{Why is \LaTeX?}

\begin{itemize}
	\item For three, technical documents drafted in \LaTeX\ just look nicer and more professional
	\item Drafting assignments using \LaTeX\ will produce work that you can be proud of
	\item It will also impress your instructor and grader
\end{itemize}

\end{frame}

%========================================================================
%========================================================================
%========================================================================
\section{Crash Course}

%========================================================================
\begin{frame}{Minimal Working Example}

\begin{itemize}
	\item We will start by creating a minimal working example of a \TeX\ document that can be compiled into a PDF
	\item We will see some important formatting options and structures
	\item Then we will jump into a complete document for a broader feature tour
\end{itemize}

\end{frame}

%========================================================================
\begin{frame}{Minimal Working Example}

Tasks:
\begin{enumerate}
	\item Create a PDF with text
	\item Adjust the margins to be 1"
	\item Add a title, author, and date
	\item Add section numbers and references
	\item Add some equations and references
\end{enumerate}

\end{frame}

%========================================================================
\begin{frame}{Some Basic Lessons}

\begin{itemize}
	\item Every document must begin with a \textit{Preamble} that defines things like:
	\begin{itemize}
		\item Document class (e.g.~article)
		\item Packages to import (with options)
		\item Formatting options, defining macros, etc.
	\end{itemize}
	\item The main body goes within a \textit{document} environment, between {\tt \textbackslash begin\{document\}} and {\tt \textbackslash end\{document\}}
\end{itemize}

\end{frame}

%========================================================================
\begin{frame}{Some Basic Lessons}

\begin{itemize}
	\item Inline math text goes between dollar signs {\tt \$\textbackslash frac\{d\}\{dx\} x\^{}2 = 2x\$}
	\item Full-sized math text goes between two dollar signs {\tt \$\$\textbackslash frac\{d\}\{dx\} x\^{}2 = 2x\$\$} or within an \textit{align} environment
	\begin{align*}
		\frac{d}{dx} x^2 &= 2x \\
		a^2 + b^2 &= c^2
	\end{align*}
	\item Lines can be commented out using the percent sign \texttt{\%}
\end{itemize}

\end{frame}

%========================================================================
\begin{frame}{Some Basic Lessons}

\begin{itemize}
	\item Any numbered item (section, subsection, equation, figure, table,~etc.) can be given a label with {\tt \textbackslash label\{mylabel\}}
	\item Its number can then be referenced elsewhere with {\tt \textbackslash ref\{mylabel\}}
	\item A star~(\texttt{*}) can be included in most of these environment names to suppress numbering
\end{itemize}

\end{frame}

%========================================================================
\begin{frame}{A Complete Document}

\begin{itemize}
	\item Next we'll take a look at a complete document to show off a variety of common technical writing features
	\item This is a set of supplementary lecture notes I posted for my Optimization Theory class earlier this semester
	\item You can find the source code on the GitHub repo
	
	\url{https://github.com/adam-rumpf/latex-examples-sp24}
\end{itemize}

\end{frame}

%========================================================================
\begin{frame}{Macros}

\begin{itemize}
	\item You can define your own macros in the preamble with {\tt \textbackslash newcommand\{\textbackslash commandname\}\{output\}}
	\item This can be useful to make shortcuts for common pieces of notation
\end{itemize}

\end{frame}

%========================================================================
\begin{frame}{\textit{Text} \textbf{Formatting}}

\begin{itemize}
	\item \textit{Italic:} {\tt \textbackslash textit\{my text\}}
	\item \textbf{Boldface:} {\tt \textbackslash textbf\{my text\}}
	\item \underline{Underlined:} {\tt \textbackslash underline\{my text\}}
	\item \textsc{Small Caps:} {\tt \textbackslash textsc\{my text\}}
	\item \textrm{Roman (Serif):} {\tt \textbackslash textrm\{my text\}}
	\item \textsf{Sans Serif:} {\tt \textbackslash textsf\{my text\}}
	\item \texttt{Typewritten:} {\tt \textbackslash texttt\{my text\}}
\end{itemize}

\end{frame}

%========================================================================
\begin{frame}{Systems of Equations}

\begin{itemize}
	\item Systems of equations can be typeset using the \texttt{align} (for single column) or \texttt{alignat} (for multiple columns) environments
	\item Ampersands~({\tt \&}) within each line will be aligned with each other, which can be used to align equals signs~({\tt =}) in systems of equations
	\item The newline character is a double backslash~({\tt \textbackslash \textbackslash})
\end{itemize}

\end{frame}

%========================================================================
\begin{frame}{Tables and Figures}

\begin{itemize}
	\item Tables and figures can be inserted using a similar format
	\item They are both types of \textit{floats}, meaning that (unless specified) \LaTeX\ will attempt to place them wherever they disrupt the text the least (rather than where they actually appear in the source text)
	\item Both can be given captions and numbers, and can be referenced
\end{itemize}

\end{frame}

%========================================================================
\begin{frame}{Tables and Figures}

\begin{itemize}
	\item Tables must be defined in the source text using the \texttt{tabular} environment, which specifies the number and alignment of the columns
	\item Figures are loaded from file references
	\item It is common to include a \texttt{figures/} folder in the project directory to contain all graphics
\end{itemize}

\end{frame}

%========================================================================
\begin{frame}{Theorems and Proofs}

\begin{itemize}
	\item Packages are available for defining and using \texttt{theorem} and \texttt{proof} environments
	\item Theorems (and lemmas, corollaries, observations,~etc.) are automatically numbered and can be referenced
\end{itemize}

\end{frame}

%========================================================================
\begin{frame}{References}

\begin{itemize}
	\item Quick references and other side notes can be included with the {\tt \textbackslash footnote} command
	\item However most citations are handled using a bibliography file and the {\tt \textbackslash cite} command
\end{itemize}

\end{frame}

%========================================================================
\begin{frame}{\textsc{Bib}\TeX}

\begin{itemize}
	\item \textsc{Bib}\TeX\ is a bibliography file format recognized by \LaTeX
	\item You can include a \texttt{.bib} file containing citation information for all books, articles, etc.~that you wish to cite in your paper
	\item Most online journal articles include links to automatically export the citation info in \textsc{Bib}\TeX\ format
\end{itemize}

\end{frame}

%========================================================================
\begin{frame}{\textsc{Bib}\TeX}

\begin{itemize}
	\item You do not need to format the bibliography entries yourself; you simply fill in tags with relevant info (author names, article title, journal, date,~etc.)
	\item Within a \TeX\ document you can then import the \texttt{.bib} file and choose a citation format (IEEE, MLA, APA,~etc.)
	\item Your paper's bibliography will then automatically populate with all works cited by the \texttt{\textbackslash cite} command within your paper
	\item All inline references will update if the bibliography changes
\end{itemize}

\end{frame}

%========================================================================
\begin{frame}{References}

{\tt \small
@article\{armijo1966,\\
author=\{Larry Armijo\},\\
title=\{Minimization of functions having \{L\}ipschitz continuous first partial derivatives\},\\
journal=\{Pacific Journal of Mathematics\},\\
year=\{1966\},\\
volume=\{16\},\\
number=\{1\},\\
pages=\{1--3\},\\
doi=\{10.2140/pjm.1966.16.1\},\\
\}
}

\vfill

L.~Armijo. \textit{Minimization of functions having Lipschitz continuous first partial derivatives}. Pacific Journal of Mathematics, 16(1):1--3, 1966. \texttt{doi:\href{https://doi.org/10.2140/pjm.1966.16.1}{10.2140/pjm.1966.16.1}}

\end{frame}

%========================================================================
\begin{frame}{Bullets and Numbered Lists}

\begin{itemize}
	\item Bullet-point lists are created with the \texttt{itemize} environment
	\item This is one of those
	\begin{itemize}
		\item It can also include sub-lists
		\item This is one of those
	\end{itemize}
\end{itemize}

\end{frame}

%========================================================================
\begin{frame}{Bullets and Numbered Lists}

\begin{enumerate}
	\item Numbered lists use the \texttt{enumerate} environment instead
	\item This is one of those
	\begin{enumerate}[a]
		\item It can also include sub-lists
		\item This is one of those
	\end{enumerate}
\end{enumerate}

\end{frame}

%========================================================================
\begin{frame}{!`D\'i\"a\c{c}r\^it\`ic\~a\l{} M\aa \u{r}k\v{s}!}

\begin{itemize}
	\item \texttt{Paul Erd\textbackslash H\{o\}s used L'H\textbackslash\^{}opital's Rule to prove H\textbackslash"older's Inequality. He celebrated with 20~\textbackslash AA\$\^{}3\$ of cura\textbackslash c\{c\}ao.}
	\item Paul Erd\H{o}s used L'H\^opital's Rule to prove H\"older's Inequality. He celebrated with 20~\AA$^3$ of cura\c{c}ao.
\end{itemize}

\end{frame}

%========================================================================
\begin{frame}{Modular Documents}

\begin{itemize}
	\item When you begin to write very large documents (like multi-section papers and books) it may be useful to separate your \TeX\ file into several smaller ones
	\item Sub-files can be imported into a master file using the \texttt{\textbackslash input} command
\end{itemize}

\end{frame}

%========================================================================
%========================================================================
%========================================================================
\section{Helpful Resources}

%========================================================================
\begin{frame}{Finding Help}

\begin{itemize}
	\item \LaTeX\ is quite widely used, and help is very readily available
	\item You can usually find answers to questions very quickly by just searching for them
	\item Overleaf, itself, has lots of useful guides
	\item Also check out \url{https://tex.stackexchange.com/}
\end{itemize}

\end{frame}

%========================================================================
\begin{frame}{Finding Help}

\begin{itemize}
	\item Sometimes you might need to read the documentation for a package you're using, or to search for a particular package
	\item All the packages that are automatically downloaded by Overleaf (or MiKTeX or TeX Live) come from CTAN: The Comprehensive \TeX\ Archive Network (\url{ctan.org})
\end{itemize}

\end{frame}

%========================================================================
\begin{frame}{The Not So Short Introduction}

\begin{itemize}
	\item \textit{The Not So Short Introduction to \LaTeX} is an extremely helpful comprehensive guide that I use often
	\item Try skimming through it to see brief rundowns of basic structures
	\item I especially recommend the \textit{List of Mathematical Symbols} section
	\item \url{https://tug.ctan.org/info/lshort/english/lshort.pdf}
\end{itemize}

\end{frame}

%========================================================================
\begin{frame}{Detexify}

\begin{itemize}
	\item \textit{Detexify} is a great online and mobile tool for finding commands for symbols you might not remember
	\item Sketch a symbol by hand, and Detexify will use machine learning to guess what symbol you're drawing
	\item It provides commands and their necessary packages
	\item \url{https://detexify.kirelabs.org/classify.html}
\end{itemize}

\end{frame}

%========================================================================
%========================================================================
%========================================================================
\section{Conclusion}

%========================================================================
\begin{frame}{90\% Perspiration}

\begin{itemize}
	\item Today's crash course could only ever be a brief feature tour
	\item Now that you've seen some things that \LaTeX\ can do, you can go out to try it yourself and to look up help if you get stuck
\end{itemize}

\end{frame}

%========================================================================
\begin{frame}{90\% Perspiration}

\begin{itemize}
	\item Learning \LaTeX\ is hard work, but it's worth it for technical professionals
	\item The best way is to simply dedicate yourself to using it regularly
	\item It will feel slow at first, but it gets easier with time and experience
\end{itemize}

\end{frame}

%========================================================================
\begin{frame}{90\% Perspiration}

\begin{itemize}
	\item I learned it by deciding to draft all of my homework and notes using \LaTeX\ upon entering grad school
	\item At first everything took way longer than writing things by hand or with \textit{Word}
	\item Now I can do most things faster in \LaTeX\ than I can with anything else
\end{itemize}

\end{frame}

%========================================================================
\begin{frame}{Thank you!}

\centering

\begin{figure}
	\includegraphics[height=0.3\textheight]{figures/broken_konigsberg_circle.png}
\end{figure}

\url{adam-rumpf.github.io}

\end{frame}

\end{document}
